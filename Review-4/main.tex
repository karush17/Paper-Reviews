           % ################################### %
           % DO NOT CHANGE THIS PART OF THE FILE%
           % ################################### %

\documentclass[12pt,letterpaper]{article}
\usepackage{fullpage}
\usepackage[top=2cm, bottom=4.5cm, left=2.5cm, right=2.5cm]{geometry}
\usepackage{amsmath,amsthm,amsfonts,amssymb,amscd}
\usepackage{lastpage}
\usepackage{enumerate}
\usepackage{fancyhdr}
\usepackage{mathrsfs}
\usepackage{xcolor}
\usepackage{graphicx}
\usepackage{listings}
\usepackage{hyperref}
\usepackage{caption}
\usepackage{subcaption}
\usepackage{float}
\hypersetup{%
  colorlinks=true,
  linkcolor=blue,
  linkbordercolor={0 0 1}
}
\renewcommand\lstlistingname{Algorithm}
\renewcommand\lstlistlistingname{Algorithms}
\def\lstlistingautorefname{Alg.}

\lstdefinestyle{Python}{
    language        = Python,
    frame           = lines, 
    basicstyle      = \footnotesize,
    keywordstyle    = \color{blue},
    stringstyle     = \color{green},
    commentstyle    = \color{red}\ttfamily
}

\setlength{\parindent}{0.0in}
\setlength{\parskip}{0.05in}
           % ################################### %
           % ################################### %
           % ################################### %



\pagestyle{fancyplain}
\headheight 35pt
                 % ENTER REVIEW NUMBER HERE %
\chead{\textbf{\large Review-4}}
           % ################################### %

\lfoot{}
\cfoot{}
\rfoot{\small\thepage}
\headsep 1.5em

\begin{document}


                 % ENTER PAPER TITLE HERE %
\begin{center}
  \large{Data-Efficient Image Recognition With Contrastive Predictive Coding}
\end{center}
           % ################################### %


Growing advancements in unsupervised learning have given rise to data-efficient algorithms which commonly fall in the regime of self-supervised learning. These methods provide improved performance in comparison to supervised learning with their representations readily transferrable to downstream tasks. The work presents an improved version of the Contrastive Predictive Coding method (CPCv2) which makes use of latent representations for image recognition in a data-efficient manner. In comparison to CPCv1, CPCv2 demonstrates improved performance as a result of 7 improvements carried out to the training setting. These improvements aid in data-efficient learning when trained on 1\% of ImageNet dataset. Furthermore, representations learnt by the model are finetuned to for efficient object-detection on the PASCAL-VOC 2007 dataset.



\end{document}
