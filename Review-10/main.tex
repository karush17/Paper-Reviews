
\documentclass[11pt,letterpaper]{article}
\usepackage{fullpage}
\usepackage[top=2cm, bottom=4.5cm, left=2.5cm, right=2.5cm]{geometry}
\usepackage{amsmath,amsthm,amsfonts,amssymb,amscd}
\usepackage{lastpage}
\usepackage{enumerate}
\usepackage{fancyhdr}
\usepackage{mathrsfs}
\usepackage{xcolor}
\usepackage{graphicx}
\usepackage{listings}
\usepackage{hyperref}
\usepackage{caption}
\usepackage{subcaption}
\usepackage{float}
\hypersetup{%
  colorlinks=true,
  linkcolor=blue,
  linkbordercolor={0 0 1}
}
\renewcommand\lstlistingname{Algorithm}
\renewcommand\lstlistlistingname{Algorithms}
\def\lstlistingautorefname{Alg.}

\lstdefinestyle{Python}{
    language        = Python,
    frame           = lines, 
    basicstyle      = \footnotesize,
    keywordstyle    = \color{blue},
    stringstyle     = \color{green},
    commentstyle    = \color{red}\ttfamily
}

\setlength{\parindent}{0.0in}
\setlength{\parskip}{0.05in}



\pagestyle{fancyplain}
\headheight 35pt
                 % ENTER REVIEW NUMBER HERE %
\chead{\textbf{\large Review-10}}
           % ################################### %

\lfoot{}
\cfoot{}
\rfoot{\small\thepage}
\headsep 1.5em

\begin{document}


                 % ENTER PAPER TITLE HERE %
\begin{center}
  \large{Counterfactual Data Augmentation using Locally Factored Dynamics}
\end{center}
           % ################################### %

Various dynamic processes in robot control consist of subprocesses which interact with each other during the execution of the global process. Sparse interactions between subprocesses allows the formulation of locally independent mechanisms for globally optimal behavior. The work proposes a novel model-free data augmentation framework by conditioning on locally factored dynamics which aid in the generation of causally valid counterfactual experiences. Counterfactual Data Augmentation (CoDA) leverages causal independence in interactions to factor local dynamics in an object-oriented manner. Improved performance of model-free agents trained on counterfactual experiences validates the suitability of CoDA in batch-constrained and goal-conditioned settings. 


\end{document}
