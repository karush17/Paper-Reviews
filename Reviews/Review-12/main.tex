
\documentclass[11pt,letterpaper]{article}
\usepackage{fullpage}
\usepackage[top=2cm, bottom=4.5cm, left=2.5cm, right=2.5cm]{geometry}
\usepackage{amsmath,amsthm,amsfonts,amssymb,amscd}
\usepackage{lastpage}
\usepackage{enumerate}
\usepackage{fancyhdr}
\usepackage{mathrsfs}
\usepackage{xcolor}
\usepackage{graphicx}
\usepackage{listings}
\usepackage{hyperref}
\usepackage{caption}
\usepackage{subcaption}
\usepackage{float}
\hypersetup{%
  colorlinks=true,
  linkcolor=blue,
  linkbordercolor={0 0 1}
}
\renewcommand\lstlistingname{Algorithm}
\renewcommand\lstlistlistingname{Algorithms}
\def\lstlistingautorefname{Alg.}

\lstdefinestyle{Python}{
    language        = Python,
    frame           = lines, 
    basicstyle      = \footnotesize,
    keywordstyle    = \color{blue},
    stringstyle     = \color{green},
    commentstyle    = \color{red}\ttfamily
}

\setlength{\parindent}{0.0in}
\setlength{\parskip}{0.05in}



\pagestyle{fancyplain}
\headheight 35pt
                 % ENTER REVIEW NUMBER HERE %
\chead{\textbf{\large Review-12}}
           % ################################### %

\lfoot{}
\cfoot{}
\rfoot{\small\thepage}
\headsep 1.5em

\begin{document}


                 % ENTER PAPER TITLE HERE %
\begin{center}
  \large{Cautious Adaptation For Reinforcement Learning in Safety-Critical Settings}
\end{center}
           % ################################### %

Real-world settings require a Reinforcement Learning (RL) agent to exhibit cautious yet optimal behavior. To that end, the work presents a Safety Critical Adaptation (SCA) framework which allows the agent to adapt to scenarios by executing a safe policy. The framework trains the agent in a non-safety critical simulation setting in order to adapt to safety-critical scenarios where failures penalize the agent heavily. The work further proposes a solution based on SCA framework in order to yield risk-averse policies for cautious adaptation. Cautious Adaptation in Reinforcement Learning (CARL) makes use of model-based RL to capture uncertainity and estimate risk followed by planning in safety-critical settings to avoid catastrophic states. CARL empirically demonstrates risk-averse behavior with fewer failures in comparison to RL baselines. 

In most real-world scenarios, agents require safe behavior which would enable risk-averse policies in unforeseen tasks. SCA framework allows safe adaptation in RL by pretraining the agent in a setting where safety is not a concern. Following pretraining, the agent is allowed to adapt to safety-critical settings where catastrophic behavior is heavily penalized. In order to demonstrate the suitability of safe adaptation using SCA, the work proposes CARL, which is based on the intuition that agents can estimate risk and reason about uncertainity while learning from a variety of scenarios. CARL is pretrained using an ensemble of probabilistic models on various sandbox environments where safety is not a concern. The framework is combined with PETS and PDDM but the method is generalizable to any model-based approach. Following pretraining, CARL adapts to safety-critical scenarios where it plans to select actions leading to risk-averse states. Planing is carried out using a novel reward computation scheme which assigns an action score to each planned trajectory. CARL minimizes risk using two variants. Firstly, the agent may seek high rewarding states in order to minimize the risk of low rewards. This is governed using a caution parameter. Secondly, the agent may minimize the risk of reaching catastrophic states. State risk-aversion is based on minimizing the proability of critical states serving as reward penalty. 

CARL is evaluated on 4 risk-prone scenarios wherein the agent needs to adapt by seeking risk-averse policies. While CARL depicts improved performance in comparison to baseline methods, it often fails to generalize to Out-of-Domain samples. For instance, CARL presents the maximum number of pole falls and boundary collisions in CartPole and DuckieTown tasks respectively. Sub-optimal risk aversion may be the result of an inaccurate model pretrained on non-critical safety settings. Additionally, the method does not provide intuition on the tradeoff between risk and reward while executng cautious behavior. 

CARL exhibits risk-averse policies which enable the agent to restrict visitation of catastrophic states. The work presents two new directions for future research. Firstly, pretraining the model on risk-prone tasks may improve the generalization of the agent on structurally similar domains. Secondly, most real-world safety-critical applications such as medical diagnosis involve policies robust to external perturbations. The framework can be further extended towards such tasks by perturbing agent's optimal behavior such as intermittent jerks during its execution. 

Risk-averse policies yield safe adaptation to unforeseen tasks. Motivated by this intuition, the work proposes SCA framework which allows RL agents to adapt to safet-critical settings by pretraining in environments where safety is not a concern. Suitability of SCA is demonstrated using CARL, which pretrains the PETS agent using model-based RL in order to yield cautious behavior in 4 safety-critical tasks. CARL achieves high rewards while adapting risk-averse policies. 



\end{document}
