
\documentclass[11pt,letterpaper]{article}
\usepackage{fullpage}
\usepackage[top=2cm, bottom=4.5cm, left=2.5cm, right=2.5cm]{geometry}
\usepackage{amsmath,amsthm,amsfonts,amssymb,amscd}
\usepackage{lastpage}
\usepackage{enumerate}
\usepackage{fancyhdr}
\usepackage{mathrsfs}
\usepackage{xcolor}
\usepackage{graphicx}
\usepackage{listings}
\usepackage{hyperref}
\usepackage{caption}
\usepackage{subcaption}
\usepackage{float}
\hypersetup{%
  colorlinks=true,
  linkcolor=blue,
  linkbordercolor={0 0 1}
}
\renewcommand\lstlistingname{Algorithm}
\renewcommand\lstlistlistingname{Algorithms}
\def\lstlistingautorefname{Alg.}

\lstdefinestyle{Python}{
    language        = Python,
    frame           = lines, 
    basicstyle      = \footnotesize,
    keywordstyle    = \color{blue},
    stringstyle     = \color{green},
    commentstyle    = \color{red}\ttfamily
}

\setlength{\parindent}{0.0in}
\setlength{\parskip}{0.05in}



\pagestyle{fancyplain}
\headheight 35pt
                 % ENTER REVIEW NUMBER HERE %
\chead{\textbf{\large Review-9}}
           % ################################### %

\lfoot{}
\cfoot{}
\rfoot{\small\thepage}
\headsep 1.5em

\begin{document}


                 % ENTER PAPER TITLE HERE %
\begin{center}
  \large{Learning a Contact-Adaptive Controller for Robust,
  Efficient Legged Locomotion}
\end{center}
           % ################################### %

Robust controllers enable efficient qudruped locomotion which requires execution various complex gaits. The work proposes a hierarchical framework for robust control of a quadruped. The hierarchical system consists of a high level controller which makes use of Reinforcement Learning (RL) to select primitives. The low level controller learns contact adaptive behavior based on these primitives using Quadrtic Programming (QP). The end-result of hierarchical framework is a robust and energy-efficient physical robot controller which generalizes to tasks not seen during training.  

The work aims to tackle robustness in real-world controllers by unifying RL with a control-based approach. The high level controller makes use of an RL policy which is based on a variant of Double Q-learning. The policy learns to select primitve gaits which consist of swing and stand states defining the contact configuration of the four feet. This eliminates the need for computationally expensive model-based techniques which require the controller to build a model of contact states. Primitives selected by the high-level policy are used by the low-level controller which models torque control as a QP problem. The controller solves the QP for foot forces based on contact constraints, base pose dynamics and swing foot control heuristics. Swing and foot control give rise to foot forces which are converted to motor torques using feet positions Jacobian matrix. The framework is trained on a number of scenarios with varying speed of the moving surface, combinations of moving and static surfaces and different orientations of the quadruped. This results in energy-efficiency and generalization of the robot to unseen tasks such as frictionless surface.


\end{document}
