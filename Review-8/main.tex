
\documentclass[11pt,letterpaper]{article}
\usepackage{fullpage}
\usepackage[top=2cm, bottom=4.5cm, left=2.5cm, right=2.5cm]{geometry}
\usepackage{amsmath,amsthm,amsfonts,amssymb,amscd}
\usepackage{lastpage}
\usepackage{enumerate}
\usepackage{fancyhdr}
\usepackage{mathrsfs}
\usepackage{xcolor}
\usepackage{graphicx}
\usepackage{listings}
\usepackage{hyperref}
\usepackage{caption}
\usepackage{subcaption}
\usepackage{float}
\hypersetup{%
  colorlinks=true,
  linkcolor=blue,
  linkbordercolor={0 0 1}
}
\renewcommand\lstlistingname{Algorithm}
\renewcommand\lstlistlistingname{Algorithms}
\def\lstlistingautorefname{Alg.}

\lstdefinestyle{Python}{
    language        = Python,
    frame           = lines, 
    basicstyle      = \footnotesize,
    keywordstyle    = \color{blue},
    stringstyle     = \color{green},
    commentstyle    = \color{red}\ttfamily
}

\setlength{\parindent}{0.0in}
\setlength{\parskip}{0.05in}



\pagestyle{fancyplain}
\headheight 35pt
                 % ENTER REVIEW NUMBER HERE %
\chead{\textbf{\large Review-8}}
           % ################################### %

\lfoot{}
\cfoot{}
\rfoot{\small\thepage}
\headsep 1.5em

\begin{document}


                 % ENTER PAPER TITLE HERE %
\begin{center}
  \large{Hindsight Credit Assignment}
\end{center}
           % ################################### %

Assigning credit to past decisions has been a challenging problem due to high variance and bootstrapped estimates. The work addresses this open problem by assigning credit to past decisions in hindsight. More specifically, the work aims to answer the question \textit{"given a state $x$, how does choosing an action $a$ affect the returns?"}. Taking into account the four main hindrances for efficient credit assignment, the work proposes a novel Hindsight Credit Assignment (HCA) scheme based on conditioned future states (HCA$|$State) and future returns (HCA$|$Return). The proposed HCA scheme is found to be efficient in assigning credit to past actions in comparison to policy gradient.

The significance of an action in a given trajectory is challenging to estimate. High variance in value estimates yields randomness in trajectories. In the case of partial observability, temporal difference learning is hindered by bootstrapped estimates which lead to inaccurate approximations. Additionally, a single trajectory may not incorporate information about all actions. To that end, the novel HCA scheme consists of future conditionals which weigh the importance of an action with respect to agent's policy. (HCA$|$State) consists of a state-conditional hindsight distribution which quantifies the relevance of a past action to a future state. Similarly, (HCA$|$Return) comprises of the reward-conditional hindsight distribution which conditions actions on future rewards. Hindsight distributions in both schemes can be generalized to time-independent distributions which yield the probability of taking an action in a some future state. 

The work compares HCA to policy gradient on three distinct tasks highlighting the problem of credit assignment in RL. HCA outpeforms policy gradient in policy adaptation and reducing variance. Additionally, HCA demonstrates suitability in learning without a temporal component. However, the scheme presents two shortcomings. Firstly, (HCA$|$State) and (HCA$|$Return) depict equivalent performance in noise reduction, indicating that the significance of their individual components has little effect on agent's actions. Secondly, (HCA$|$Return) depicts comparable performance to the baseline policy gradient in the delayed effect task as a result of modeling the conditional noise distribution. This presents a hindrance in scalability of the algorithm to large action spaces wherein the agent may have a variety of options to select in its initial state. 

While the policy gradient algorithm assign credit based on noisy reward distributions, HCA makes use of an efficient future conditional distribution to assess the relevance of actions. The scheme opens several new directions for future work with two of them being the scalability of hindsight distributions and their generalization to real-world tasks. Evaluating the scalability of HCA to large action spaces and sparse future rewards is an interesting direction for future work as these scenarios are themselves hindered by efficient credit assignment. On the other hand, generalizing HCA to real-world tasks with variant dynamics such as robotic control will yield insights into the relevance and challenges of learning hindsight distributions. 

The work tackles the problem of efficient credit assignment by proposing the novel HCA scheme. HCA learns hindsight distributions conditioned on future states and rewards from the agent's trajectories. HCA conditionals are used in importance weights with the agent's original policy. This helps in exactly assessing the relevance of an action in a given state. Additionally, hindsight distributions are made temporally-independent in order to yield estimates which are informative of past actions in any timestep of the trajectory. HCA demonstrates improved policy adaptation and noise reduction in partially-observable and challenging noisy dynamics. Suitability of learning hindsight distributions opens new avenues towards their scalability and practical relevance. 

\end{document}
