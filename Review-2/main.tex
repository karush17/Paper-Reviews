           % ################################### %
           % DO NOT CHANGE THIS PART OF THE FILE%
           % ################################### %

\documentclass[11pt,letterpaper]{article}
\usepackage{fullpage}
\usepackage[top=2cm, bottom=4.5cm, left=2.5cm, right=2.5cm]{geometry}
\usepackage{amsmath,amsthm,amsfonts,amssymb,amscd}
\usepackage{lastpage}
\usepackage{enumerate}
\usepackage{fancyhdr}
\usepackage{mathrsfs}
\usepackage{xcolor}
\usepackage{graphicx}
\usepackage{listings}
\usepackage{hyperref}
\usepackage{caption}
\usepackage{subcaption}
\usepackage{float}
\hypersetup{%
  colorlinks=true,
  linkcolor=blue,
  linkbordercolor={0 0 1}
}
\renewcommand\lstlistingname{Algorithm}
\renewcommand\lstlistlistingname{Algorithms}
\def\lstlistingautorefname{Alg.}

\lstdefinestyle{Python}{
    language        = Python,
    frame           = lines, 
    basicstyle      = \footnotesize,
    keywordstyle    = \color{blue},
    stringstyle     = \color{green},
    commentstyle    = \color{red}\ttfamily
}

\setlength{\parindent}{0.0in}
\setlength{\parskip}{0.05in}
           % ################################### %
           % ################################### %
           % ################################### %



\pagestyle{fancyplain}
\headheight 35pt
                 % ENTER REVIEW NUMBER HERE %
\chead{\textbf{\large Review-2}}
           % ################################### %

\lfoot{}
\cfoot{}
\rfoot{\small\thepage}
\headsep 1.5em

\begin{document}


                 % ENTER PAPER TITLE HERE %
\begin{center}
  \large{Momentum Contrast for Unsupervised Visual Representation Learning}
\end{center}
           % ################################### %

Unsupervised learning has seen a tremendous growth in the development of visual tasks. Various unsupervised methods in deep learning consist of self-supervision, or contrastive learning, wherein the loss function is defined on a pretext task consisting of true labels as intrinsically generated entities. This aids in efficient recognition of images from large datasets and successfully transfer the learned features to downstream tasks such as object detection. The work presents a contrastive learning algorithm called Momentum Contrast (MoCo). MoCo models contrastive learning as a dictionary-lookup task with the query being the input to the model and the keys being augmentations of the input image. The encoded query is contrasted against the momentum-encoded keys from a queue which aid in learning rich representations that are transferrable to downstream tasks. MoCo is shown to outperform various supervised learning algorithms on the ImageNet dataset and 7 detection tasks from other large-scale bechmarks.

The MoCo algorithm comprises of a query encoder and a key encoder. The query encoder encodes the visual query and compares it against the encoded keys using the similarity dot product. Keys are generated from augmentations of the input image and encoded using a momentum encoder which is a slow moving average of the query encoder. Utilizing momentum-based updates leads to consistency in keys of the dictionary which has proven to be a hindrance in the memory-bank approach and previous works on contrastive learning. Moreover, MoCo makes use of a queue for comparing dynamically generated keys with the query. The queue leads to the construction of a larger dictionary in comparison to the simpler end-to-end method which can only accomodate smaller batch-sizes due to memory constraints. Additionally, MoCo makes use of shuffling batch-normalization which shuffles the sample order of the mini-batch before distributional execution. Shuffling batch-normalization prevents the leakage of information in comparison to its standard counterpart in the ResNet architecture.  

MoCo generalizes on a number of benchmarks including ImageNet-1M (IM-1M) and Instagram-1B (IG-1B). Representations learned as a result of the momentum updates are readily transferrable to downstream object detection tasks on the PASCAL-VOC, COCO and various other datasets. Moreover, the ablation study presented on the various constituents of MoCo highlight the effectiveness of the proposed approach. However, the MoCo algorithm makes use of only a single pretext task (instance discrimination as dictionary lookup) and does not yield any insights into what kind of other pretext tasks may be used to improve the performance of the model. Additionally, MoCo is able to demonstrate only incremental gains from IM-1M to IG-1B given that the addition of data is significant. These gains highlight the need for proper exploitation of large-scale datasets.


\end{document}
