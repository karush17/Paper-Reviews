           % ################################### %
           % DO NOT CHANGE THIS PART OF THE FILE%
           % ################################### %

\documentclass[11pt,letterpaper]{article}
\usepackage{fullpage}
\usepackage[top=2cm, bottom=4.5cm, left=2.5cm, right=2.5cm]{geometry}
\usepackage{amsmath,amsthm,amsfonts,amssymb,amscd}
\usepackage{lastpage}
\usepackage{enumerate}
\usepackage{fancyhdr}
\usepackage{mathrsfs}
\usepackage{xcolor}
\usepackage{graphicx}
\usepackage{listings}
\usepackage{hyperref}
\usepackage{caption}
\usepackage{subcaption}
\usepackage{float}
\hypersetup{%
  colorlinks=true,
  linkcolor=blue,
  linkbordercolor={0 0 1}
}
\renewcommand\lstlistingname{Algorithm}
\renewcommand\lstlistlistingname{Algorithms}
\def\lstlistingautorefname{Alg.}

\lstdefinestyle{Python}{
    language        = Python,
    frame           = lines, 
    basicstyle      = \footnotesize,
    keywordstyle    = \color{blue},
    stringstyle     = \color{green},
    commentstyle    = \color{red}\ttfamily
}

\setlength{\parindent}{0.0in}
\setlength{\parskip}{0.05in}
           % ################################### %
           % ################################### %
           % ################################### %



\pagestyle{fancyplain}
\headheight 35pt
                 % ENTER REVIEW NUMBER HERE %
\chead{\textbf{\large Review-5}}
           % ################################### %

\lfoot{}
\cfoot{}
\rfoot{\small\thepage}
\headsep 1.5em

\begin{document}


                 % ENTER PAPER TITLE HERE %
\begin{center}
  \large{A Learning Algorithm for Boltzmann Machines}
\end{center}
           % ################################### %


Increasing interest in parallel search and connection-based methods has motivated efficient learning algorithms. Learners aim to represent the intricate aspects of data by communicating the compatibility of entities within themselves. One such learner is the Boltzmann Machine which is trained using an energy-based objective. The work presents a novel algorithm for training Boltzmann Machines which is based on energy compatibilities between the input and output representations. Learning is carried out in an architecture consisting of visible and hidden computational units utilizing a Boltzmann distribution to predict the probabilities of spatial representations. These probabilities are collected towars the end of each trial and used in the update rule for parameters of the network. The network is tested on a number of bit encoding tasks with growing levels of difficulty. 



\end{document}
